% LaTeX file for resume 
% This file uses the resume document class (res.cls)

\documentclass{res} 
%\usepackage{helvetica} % uses helvetica postscript font (download helvetica.sty)
%\usepackage{newcent}   % uses new century schoolbook postscript font 
\setlength{\textheight}{9.5in} % increase text height to fit on 1-page 

\begin{document} 

\name{PER KARLSSON\\[12pt]}     % the \\[12pt] adds a blank
				        % line after name      

\address{\bf  PRESENT ADDRESS\\995 36th St\\Emeryville, CA 94608\\+1 650 283-7222}
\address{\bf PERMANENT ADDRESS \\ Bonared Vastergarden 6 \\51198 Hyssna, Sweden \\ +46 320 39261}

\begin{resume}      

\section{EDUCATION}
    B.Sc in Computer Science and Engineering, GPA 4.0/4.0\\ \emph{Linkoping University}, 2014 \\  \\
    M.Sc in Computer Science and Engineering, \\ \emph{Linkoping University} and \emph{Stanford University}. Expected to graduate September 2014. GPA 3.9/4.0
 
\section{EXPERIENCE}
   \vspace{-0.1in}	
   \begin{tabbing}
   \hspace{2.3in}\= \hspace{2.4in}\= \kill % set up two tab positions
    {\bf Technical Director} \>Pixar Animation Studios     \>Summer 2012-Present\\
                             \>Emeryville, CA
   \end{tabbing}\vspace{-20pt}      % suppress blank line after tabbing
    Co-authored the rigid body simulation pipeline built on top of existing \emph{bullet} and \emph{Physbam} rigid body libraries, developing the C++ backend and the python wrapped frontend. This system is used for all rigid body simulations in the Pixar inhouse animation system.

   In production, worked on the feature movie \emph{The Good Dinosaur} where I led the vegetation simulation effort in the context of character interaction. Also worked on the movie \emph{Inside Out} where I authored a new tool for animators to simulate secondary motion on top of hand-animated primary motion, saving them the time and effort of having to manually animate such effects.

   \begin{tabbing}
   \hspace{2.3in}\= \hspace{2.9in}\= \kill % set up two tab positions
    {\bf Software Engineer} \>Naiad Group at Exotic Matter \> Summer 2011\\
                          \>Stockholm, Sweden
   \end{tabbing}\vspace{-20pt}
    Developed the Naiad Ocean Toolkit, a frequency spectrum based wave generator. The wave simulation generated by the toolkit can be merged with an existing Naiad FLIP water simulation, enabling visual effects studios to exclusively use Naiad for ocean sequences. I also wrote 3rd party Naiad plugins for Houdini and the Arnold renderer, allowing studios to use Naiad in their pipelines without any additional development work.
   \begin{tabbing}%
   \hspace{2.3in}\= \hspace{3.1in}\= \kill % set up two tab positions          
   {\bf Teaching Assistant}  \>Linkoping University. \> 2009-2010\\
                          \>Norrkoping, Sweden
   \end{tabbing}\vspace{-20pt}
    Lecturer in single variable calculus and linear algebra. Grading assignments.       


\section{SKILLS}
Extensive knowledge of simulation, parallel computing, real-time performance and offline rendering.
Proficient programming skills in C/C++, Python, CUDA, GLSL, Matlab.
 
\section{HONORS AND AWARDS}
{\bf1st Prize in the Rendering Competition}, \emph{Stanford cs348b}, 2012\\
Global illumation model for rendering foam and splashes.

{\bf Most Technical Advanced Game Award}, \emph{Stanford cs248}, 2012\\
GPU based Ocean Waves, Deferred shading with Depth of
Field, Motion Blur and Bloom.

{\bf Scholarships}
 \vspace{-0.2in}	
\begin{tabbing}%
   \hspace{3.3in}\= \hspace{3.6in}\= \kill % set up two tab positions          
   ~~~~~~$\star$ Stanford Tuition Fees \& Housing \> $\star$ SIGGRAPH Conference Los Angeles \\
   ~~~~~~$\star$ Norrkopings Polytekniska stipendium \> $\star$ Hans-Olof Johanssons stipendiefond
   \end{tabbing}\vspace{-20pt}
\end{resume}
\end{document}
